\documentclass[a4paper]{article}

\usepackage{graphicx}
\usepackage{xcolor}
\usepackage{listings}
\graphicspath{ {./} }
\definecolor{codegreen}{rgb}{0,0.6,0}
\definecolor{codegray}{rgb}{0.5,0.5,0.5}
\definecolor{codepurple}{rgb}{0.58,0,0.82}
\definecolor{backcolour}{rgb}{0.95,0.95,0.92}
\lstdefinestyle{mystyle}{
    backgroundcolor=\color{backcolour},   
    commentstyle=\color{codegreen},
    keywordstyle=\color{magenta},
    numberstyle=\tiny\color{codegray},
    stringstyle=\color{codepurple},
    basicstyle=\ttfamily\footnotesize,
    breakatwhitespace=false,         
    breaklines=true,                 
    captionpos=b,                    
    keepspaces=true,                 
    numbers=left,                    
    numbersep=5pt,                  
    showspaces=false,                
    showstringspaces=false,
    showtabs=false,                  
    tabsize=2
}

\lstset{style=mystyle}

\usepackage[hmargin=1.5cm,vmargin=1.5cm]{geometry}

\usepackage{polski}
\usepackage[utf8]{inputenc}

\usepackage{hyperref}
\hypersetup{colorlinks=true,linkcolor=blue}

\title{Nauka C\#}
\author{Krystian Gronkowski}
\date{\today}

\begin{document}

\maketitle
\tableofcontents

\pagebreak
\section{Hello, world!}
\subsection{Konfigurowanie kompilatora}
Zanim zaczniemy programować, zaintalujmy kompiler, aby móc otworzyć program który napiszemy.
\\\\W Linuxie mcs można zainstalować za pomocą komend:
\\\\sudo apt-get update
\\sudo apt-get install mono-mcs
\\\\Natomiast w Windowsie należy najpierw zainstalować .NET Framework, a potem dodać ścieżkę instalacji w zmiennej środowiskowej "PATH"
\\\\Alternatywnie, jeżeli ktoś nie chce instalować kompilera, można użyć jednego z wielu edytorów C\# online, np: \url{https://www.onlinegdb.com/online_csharp_compiler}
\subsection{Console.WriteLine}
Nadszedł czas na stworzenie pierwszego programu!
\\Struktura piku źródłowego C\# wygląda następująco:
\lstset{language=C}
\begin{lstlisting}[frame=single]
using System;

class NazwaPliku {
  static void Main() {
    //Kod
  }
}
\end{lstlisting}
\subsection{Console.ReadLine}
\section{Operowanie na danych}
\section{Tablice i listy}
\section{Losowanie liczb}
\pagebreak
\begin{thebibliography}{9}
\bibitem{texbook}
https://automatykanacodzien.pl/2020/07/24/kompilacja-i-uruchomienie-kodu-c-w-linii-polecen-windows-linux/
\end{document}